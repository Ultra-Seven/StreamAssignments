%!TEX root = ../main.tex

\section{Introduction}

\label{s:intro}


% In spite of the growing importance of big data, sensors, and automated data collection, manual data entry continues to be a primary source of high-value data across organizations of all sizes, industries, and applications: sales representatives manage lead and sales data through SaaS applications~\cite{salesforce}; human resources, accounting, and finance departments manage employee and corporate information through terminal or internal browser-based applications~\cite{sap}; driver data is updated and managed by representatives throughout local DMV departments~\cite{dmv,dmvsystem}; consumer banking and investment data is managed through web or mobile-based applications~\cite{betterment,chase}. In all of these examples, the database is updated by translating form-based human inputs into INSERT, DELETE or UPDATE query parameters that run over the backend database---in essence, these are instances of OLTP applications that translate human input into stored procedure parameters. Unfortunately, numerous studies~\cite{kandel2012,krishnan2016hilda,Barchard20111834}, reports~\cite{citibank,Yates10,Grady13,Robeznieks05} and citizen journalists~\cite{iquantnyc} have consistently found evidence that human-generated data is both error-prone, and can significantly corrupt downstream data analyses~\cite{iquantnycnypd}. Thus, even if systems assume that data import pipelines are error-free, queries of human-driven applications continue to be a significant source of data errors, and there is a pressing need for solutions to diagnose and repair these errors. Consider the following representative toy example that we will use throughout this paper:

In this paper, we present \sys, which does some stuff.   

\begin{itemize}[leftmargin=*, topsep=0mm, itemsep=0mm]

\item We formally analyze the design space for prefetching-based visual data exploration systems to understand the conditions for which prefetching will be effective.

\item We design a novel client-server architecture.

\item We present a suite of optimizations that further reduce interaction response times, and show how to extend to new offline data structures, including sampling and data cubes.

\item We perform a thorough evaluation of the XXX characteristics \sys's performance.  

\end{itemize}
